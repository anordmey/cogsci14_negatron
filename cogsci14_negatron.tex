\documentclass[10pt,letterpaper]{article}

\usepackage{cogsci}
\usepackage{pslatex}
\usepackage{pdfsync}
\usepackage{apacite}
\usepackage{amsmath}
\usepackage{graphicx}
\usepackage{topcapt}
\usepackage{color}
%\usepackage{setspace}
%\singlespacing


\title{A Pragmatic Account of the Processing of Negative Sentences}
\author{{\large \bf Ann E. Nordmeyer} \\ \texttt{anordmey@stanford.edu}\\ Department of Psychology \\ Stanford University \\ 
\And {\large \bf Michael C. Frank} \\ \texttt{mcfrank@stanford.edu} \\ Department of Psychology \\ Stanford University \\ }

%\affiliation{Stanford University}

%\abstract{Put Abstract Here.}

%\vspace{3.0ex}

%\noindent Please address correspondence to: Ann E. Nordmeyer, Department of Psychology, Stanford University, 450 Serra Mall, Building 420 (Jordan Hall), Stanford, CA 94305, email: \texttt{anordmey@stanford.edu}.}

%shorttitle{FYP Proposal}
%\rightheader{FYPP}
%\leftheader{A. E. Nordmeyer}

\begin{document}
\maketitle

\begin{abstract}

ABSTRACT


\textbf{Keywords:} 
Negation; sentence processing; pragmatics
\end{abstract}

%%%%%%%%% INTRO %%%%%%%%% 
\section{Introduction}

Language is a powerful tool that allows us to communicate with others about the world around us.  Often we use language to describe the state of the world as we see it, using labels and descriptions to refer to objects and people.  Sometimes, however, we want to use language to describe the world as it is \emph{not}.  Perhaps we want to correct a false statement, or describe the absence of a characteristic.  Negation allows us to express these concepts. Although previous work on sentence comprehension has suggested that negative sentences are difficult to process \cite{hclark1972, carpenter1975, just1971, just1976}, we hear negative sentences every day and as listeners we frequently and accurately interpret these sentences in conversation.  

There is a consistent finding in the literature on negation that participants are slower to evaluate sentences containing a negative element  \cite{hclark1972, carpenter1975, just1971, just1976}.  For example, Clark and Chase (1972) asked participants to compare positive or negative sentences to simple pictures which either matched or did not match the sentences.  Participants were overall slower to respond to negative sentences.  Furthermore, they found an interaction between the type of sentence and the truth of the sentence, with participants responding fastest to true positive sentences and \emph{slowest} to true negative sentences.  This work has been taken as evidence for a propositional view of negation, in which a negative sentence is represented as the denial of an affirmative proposition, leading to greater processing time for negative as opposed to positive sentences. 

The finding that negative sentences take longer to process than positive sentences has been replicated in other paradigms as well.  Electroencephalography studies have shown that semantically unexpected/false sentences such as ``a robin is a truck'' elicit a more negative peak at the N400 compared to sentences such as ``a robin is a bird''.  However, sentences such as ``a robin is not a truck'' produce a greater N400 response than sentences such as ``a robin is not a bird'', suggesting that the negative element ``not'' is not processed immediately (\citeNP{fischler1983}, see also \citeNP{ludtke2008}).  Similar results have been found using probe-recognition tasks.  For example, the sentence ``The door was open'' primes recognition of  a matching picture (an open door) compared to a mismatching picture (a closed door) after 750 ms delay, whereas negative sentences such as ``The door was not open'' do not prime the matching picture (a closed door) until a 1000 ms delay.  (\citeNP{kaup2006}, see also \citeNP{kaup2003, hasson2006}).  Although there is some disagreement about the nature of these representations (e.g. as the denial of a proposition (see \citeNP{hclark1972}), or a mental model of the negated and actual state of affairs (see \citeNP{kaup2003}), this work collectively suggests that negative sentences are more difficult to process than positive sentences.  
  
There is a critical difference, however, between evaluating sentences presented without any context in a laboratory setting and comprehending speech in the real world. According to Grice's cooperative principle \cite{grice1975}, speakers should produce utterances that are appropriately informative, relevant, and unambiguous.  Negative sentences presented without any contextual support often violate this principle.  If my friend asks me what I did this week, and I respond ``I didn't buy a car'', my contribution to the conversation is neither relevant nor informative.  The fact that unsupported negative utterances are generally rated as more ambiguous than unsupported affirmative utterances has been documented experimentally \cite{glenberg1999}.  In general, negative utterances are produced in contexts where there is some \emph{expectation} that the speaker wishes to negate.  For example, if my friend knows that I have been in the market for a new car, then she may be asking about my week because she expects I bought a new car, and my statement ``I didn't buy a car'' becomes relevant and informative.  

Noting that denials are generally produced in response to a violation of expectations, Wason (1965) designed a study to examine whether pragmatic constraints such as context would affect how participants responded to negative sentences.  Participants viewed stimuli consisting of 8 colored dots, in which 7 dots were one color and 1 dot was a different color.  Participants were asked to describe the stimuli, and then evaluate positive and negative sentences about the stimuli.  Wason found that when participants' descriptions of the stimuli highlighted the fact that one of the dots was an exception to the rule, they were faster to evaluate negative sentences about the ``odd'' dot.  Additional work has shown that when participants read sentences that set up a supportive context, reading times for negative sentences are reduced \cite{glenberg1999}, and false positive and false negative sentences show similar N400 responses when the negative sentences are pragmatically licensed \cite{nieuwland2008}.  Children are also sensitive to context effects when processing negative sentences, responding faster to respond to negative sentences about a toy that is different from other toys in the group \cite{devilliers1975}.

Some researchers have found that some contexts are more effective than others at facilitating the processing of negative sentences.  In one such study, contexts which explicitly mentioned or strongly implied the negated characteristic were more effective at reducing reading times of negative sentences compared to contexts which did not mention the relevant characteristic \cite{ludtke2006}.  Critically, these contexts did not have a significant effect on the reading times for positive sentences.  In addition, a mouse-tracking study conducted by Dale and Duran (2011) \nocite{dale2011} found that enhanced contexts lead to significantly smoother mouse trajectories when selecting whether a sentence was true or false, but simple one-sentence contexts did not have an effect.  This work suggests that there is some graded effect of context, with some types of context having a greater effect on negation processing than others, though few studies have thoroughly examined the effect of the strength of the context.

Studies of the effects of pragmatics on linguistic processing exist in other domains as well.  For example, extensive research has been conducted on the processing of scalar implicatures, such as the inference made by listeners that the word "some" excludes the possibility "all", even though semantically the word "some" can encompass "all".  Work by Huang and Snedeker (2009, 2011) \nocite{huang2009, huang2011} suggests that there is a gap between semantic and pragmatic processing, with participants taking significantly longer to resolve referential ambiguity when the resolution involves computing a pragmatic implicature (i.e. some = not all).  However, other work using a similar paradigm suggests that these pragmatic inferences are computed rapidly \cite{grodner2010}.  Degen and Tanenhaus (cite?) propose that slow reaction times to compute the scalar implicature for ``some'' may be due to uncertainty about the "question under discussion'' (QUD) \cite{roberts1996}.  This proposal may be relevant to the different processing times seen for negative sentences.  Tian et. al (2010) \nocite{tian2010} found negative constructions of the form ``Eric didn't iron his shirt'' lead participants to respond faster to images of an ironed shirt compared to a wrinkled shirt (consistent with e.g. \citeNP{kaup2006}), negative sentences such as ``It was Eric who didn't iron his shirt'' resulted in the opposite pattern (e.g. faster responses to a wrinkled shirt).  This was interpreted as evidence that the first construction leads to the accommodation of a positive QUD (``Did Eric iron his shirt?''), while the second construction leads to the accommodation of a negative QUD (``Who didn't iron his shirt?'').  Further exploration of the ways that context effects the processing of negative sentences could further our understanding not only of negation, but of how pragmatics can influence sentence processing more generally.  

This paper examines the possibility of a pragmatic account of negation comprehension.  We begin by replicating the finding that context facilitates the processing of negative sentences in a sentence verification task, using visual contexts to set up expectations about the base rate of a certain characteristic (Study 1).  We then explore the effect of the strength of the context by varying that base rate parametrically (Studies 2a and 2b).  In order to understand how context might be having an effect on reaction times to evaluate negative sentences, we test the possibility that participants' expectations of how a speaker would describe a given picture are related to processing times (Study 3).  Finally, we attempt to model this relationship between speaker expectations and reaction time, calculating the probability that a speaker would produce an utterance given a certain context to create a model of pragmatic surprisal (see \citeNP{levy2008}, \citeNP{frank2012}).  

\section{Study 1: Context vs. No Context}
Study 1 replicates the finding that a simple visual context reduces the latency to evaluate negative sentences \cite{wason1965}.  The study was conducted using Amazon's Mechanical Turk (AMT) website, an online crowd-sourcing platform for conducting survey research, and was programmed using Javascript, allowing us to collect reaction times for participants' responses.  Previous work has 
demonstrated that AMT is an effective tool for collecting reaction time data \cite{crump2013}.  

We used a sentence verification task in which participants were asked to evaluate whether positive and negative sentences (e.g. ``Bob has/has no apples'') were true or false compared to a picture (e.g. of a person holding apples or holding nothing).  Half of the participants first viewed a context screen which showed three characters holding the named item (e.g. three boys standing with apples).  Previous work examining context effects on the processing of negative sentences required participants to actively engage with the context, either by describing the stimuli \cite{wason1965} or by reading sentences \cite{glenberg1999, ludtke2006, dale2011}.  This study explores whether a passively-viewed visual context is sufficient for facilitating the processing of negative sentences.  
\subsection{Method}

\subsubsection{Participants}
100 participants were recruited to participate in an online experiment through Amazon's Mechanical Turk website.  Participants ranged in age from 18-65, with the majority of participants being between 18 and 25 (45 participants).  63 participants were male and 37 were female.  Participation was restricted to individuals in the United States, and participants who indicated that their primary language was something other than English were excluded from analysis.  Participants were paid 30 cents to participate in the study, which took approximately 5 minutes to complete.  

\subsubsection{Stimuli}

\begin{figure}[t]
\begin{center} 
\includegraphics[width=3.25in]{figures/negatron_trialfig2.pdf}
\caption{\label{fig:trial} An example of a context slide and a trial slide (true negative). }
\vspace{-5mm}
\end{center} 
\end{figure}

28 trial items were created in which a character was shown holding either two of the same common, recognizable objects (such as apples, buckets, toy cars, cookies, flowers, kites, umbrellas, etc.), or holding nothing.  On each trial, beneath the picture, was a sentence of the form ``[NAME] has/has no [ITEM]''.  Half of the sentences were positive and half were negative, and they were paired with pictures such that half were true and half were false.  This negative construction was used to make the negative sentences match the positive sentences as closely as possible, with the only difference being the presence or absence of the word ``no''.  

The experiment was fully crossed, with participants receiving 7 true positive, 7 false positive, 7 true negative and 7 false negative sentences over the course of the study.  Trials were presented to participants in a randomized order.

Examples of the four trial types (true positive, false positive, true negative and false negative) can be seen in figure 1.    

Participants were randomly assigned to either the ``No Context'' condition or the ``Context'' condition.  Participants in the ``No Context'' condition saw a blank screen with a fixation cross in the center before each trial, while participants in the ``Context'' condition viewed a context slide before each trial.  The context slide showed a picture of three characters, each holding two items.  The characters in the context condition all differed from the trial character and from each other in hair and shirt color.  Beneath the three people was the sentence ``Look at these [boys/girls]!''  An example of a context slide and a trial slide can be seen in figure 1.  


\subsubsection{Procedure}

Participants were first presented with an instructions screen which described the task and informed them that they would be participating in a psychology experiment and could stop at any time.  Once participants accepted the task, they were given eight practice trials in which they were asked to judge whether a sentence about the color of a character's clothing was true or false.  In the practice trials, as in the actual experiment, participants first saw either a blank screen or a context screen showing three characters.  During the practice trials, if participants answered a question incorrectly a pop-up window informed them that they were incorrect and instructed them to try again.  At the end of the practice, another instructions screen reminded them of the procedure and told them to press space to start the ``game''.  

Before each trial, participants either viewed a blank screen with a fixation cross or a context screen, depending on the condition they were assigned to.  In both cases the slide instructed participants to ``Please wait...", and the experiment automatically advanced after 3 seconds.  They then saw a picture and a sentence about that picture, and they were asked to select whether the sentence was true or false by pressing ``Q'' for false or ``P'' for true.  Participants were initially instructed to  use their left hand to select the ``Q'' key and their right hand to press the ``P'' key, and to answer as quickly and accurately as possible, and to .  Once participants made their selection by pressing "P" or "Q", they were automatically brought to the next trial after a 500 ms pause.  Reaction times, measured as the time from when the pictures/sentence were presented to the moment when the participant made their response, were recorded for each trial.

At the end of the experiment, participants were taken to a screen that collected demographic information.  Participants were required to enter their gender, age, and native language, and then were invited to provide optional comments about the task.  

\subsection{Results}

%I thought I would include a table of means for reproducibility, but it ends up looking pretty dense and cluttered.  I think I would rather include this in an appendix, or post on the lab website, for anyone interested in replication; otherwise most people reading the paper will find the graphs sufficient.  
%\begin{table}[t]
%\caption{\label{tab:e1means}Means and standard deviations of reaction times in milliseconds for each of the four trial types, for each of the two context conditions.}
%\begin{center}
%\small\addtolength{\tabcolsep}{-5pt}
%\begin{tabular}{ l  r  r  r  r } 
%\hline
%& \multicolumn{2}{c}{No Context} & \multicolumn{2}{c}{Context} \\
%  Trial Type & Mean& SD & Mean & SD \\ \hline                    
%True Positive & 1517 & 351 &  1509 & 535\\
% False Positive & 1727 & 406 &  1599 & 479\\
% True Negative& 1806 & 376 & 1581 & 483\\
%  False Negative & 1719 & 317 & 1655 & 614 \\
%\hline
%\end{tabular}
%\end{center}
%\end{table}

\begin{figure}
\begin{center} 
\includegraphics[width=3.25in]{figures/study1_linegraph.pdf}
\caption{\label{fig:e1line}Reaction times for each trial type, across different conditions.  Error bars represent the 95\% confidence intervals.}
\end{center} 
\end{figure}

\begin{table}[t]
\caption{\label{tab:e1model} Coefficient estimates from a linear effects model estimating the effects of context condition, sentence type, and response on reaction time, accounting for random effects of participant and item.}
\begin{center}
\small\addtolength{\tabcolsep}{-5pt}
\begin{tabular}{ r  r  r  r  } 
\hline
 & Coefficient & Std. Err. &  $t$ value \\ \hline 
  Intercept & 1727 & 75 & 22.91 \\ 	            
Sentence Type(Negative) & -4 & 48 & -0.09\\
Truth Value(True) &  -207 & 44 & -4.70\\
Context Condition(Context) & -130 & 100 & -1.30\\
Sentence Type $\times$ Context Condition & 53 & 68 & 0.78\\
Sentence Type $\times$ Truth Value & 282 & 65 & 4.34\\
Context Condition $\times$ Truth Value &  118 & 62 & 1.91\\
Sentence $\times$ Context $\times$ Truth Value & -249 & 90 & -2.78\\
\hline
\end{tabular}
\end{center} 
\end{table}

Study 1 examined the effect of a visual context on the processing of negative sentences.  Previous research led us to expect a main effect of sentence type, with negative sentences incurring greater RTs than positive sentences, as well as a main effect of truth value, with false sentences incurring greater RTs than true sentences, and a sentence type*truth value interaction in the no context condition \cite{hclark1972}.  We hypothesized that participants in the context condition, who first viewed a visual context setting up an expectation (e.g. that boys have apples), would show faster responses to negative sentences than participants in the no context condition.  

Six participants who listed a language other than English as their native language were excluded from analysis.  Seven additional participants were excluded for having participated in a previous iteration of the experiment.  Four participants were excluded for having an overall accuracy of below 80\%.  Thus, data from a total of 83 participants were analyzed, 40 in the no context condition and 40 in the context condition.  

Only correct trials were analyzed, due to the well-documented speed-accuracy tradeoff found in measures of reaction time.  Trials with RTs greater than 3 standard deviations from the mean were also excluded from the analysis.  A graph of these data can be seen in Figure \ref{fig:e1line}.  Looking just at the data for participants in the ``no context'' condition, the effects seen here replicate previous work on the processing of negative sentences \cite{hclark1972, carpenter1975, just1971, just1976}, with a main effect of negation and an interaction between sentence type and response.  However, comparing this to the data from participants in the context condition, there appears to have been a facilitating effect of context, such that participants who first viewed the context slides had lower RTs.  This effect was strongest for true negative sentences.  

To examine the reliability of our findings, we fit a linear mixed-effects model to reaction times in response to sentences.  We examined the interaction between sentence type, truth value, and context on reaction times.    Coefficients for this model are shown in Table \ref{tab:e1model}.\footnote{All mixed-effects models were fit using the lme4 package in R version 2.15.3.  The model specification was as follows: \texttt{RT $\sim$ sentence~$\times$~truth~$\times$~context + (sentence~$\times$~truth~\textbar~subject) +  (sentence~$\times$~truth~\textbar~item)}.  Significance was calculated using the standard normal approximation to the $t$ distribution \cite{barr2013}.}  Results of this model show a main effect of truth value, with significant faster reaction times for true sentences compared to false sentences ($\beta= -207$, $p< .001$).  Although there was no main effect of negation across both conditions, there was an interaction between sentence type and truth value ($\beta= 282$, $p< .001$), replicating the finding that participants respond fastest to true positive sentences but slowest to true negative sentences.  However, there is a significant 3-way interaction between context condition, sentence type, and truth value ($\beta= -249$, $p< .01$), suggesting that this interaction is primarily driven by the slow RTs for true negative sentences in the no context condition.  

(((NOTE:  This has always been an issue, but changing the base level of the truth value variable dramatically effects the lmer model, and I'm not sure which way I should present it.  If I make FALSE the base level, there is a significant effect of negation and a significant interaction between context and negation, but this never makes sense to me because the True Negatives are more what we are interested in.  )))

\subsection{Discussion}

Study 1 provides further evidence for the facilitating effect of context on the processing of negative sentences.  Several previous studies have shown that embedding negative sentences in a supportive linguistic context leads to faster processing of negative sentences \cite{wason1965, glenberg1999, ludtke2006, dale2011}.  In this study, we presented some participants with a visual context that was designed to set up an expectation for what the trial picture might look like.  We found that viewing these contexts before each trial lead to faster processing of negative sentences, with true negative sentences showing the greatest effect.  

To understand why context had the greatest effect on true negative sentences, consider what a true negative trial looks like to a participant in the no context condition.  These are trials in which the participant has no expectation about what the character might be holding, because no context was provided to set up such an expectation.  The participant would then see a picture of a boy, not holding anything, with the sentence ``Bob has no apples''.  These types of trials likely cause participants to falter because there is no reason for ``apples'' to be mentioned at all.  However, when a participant first views a context such as the one pictured in Figure \ref{fig:trial}, they can form an expectation that boys hold apples.  Now, when the trial shows a boy with no apples, a sentence such as ``Bob has no apples'' makes sense and can be processed rapidly, because the image violates an expectation that boys should have apples.  

Study 1 contributes to a body of evidence suggesting that negative sentences are more felicitous when they are used to negate an expectation that participants might hold, and that such expectations can be set up by an appropriate context.  In Study 2, we examine how systematically manipulating the context might produce changes in reaction times by altering the strength of the expectations set up by the context.  

\section{Study 2: Varying strength of context}
Study 1 supported the finding that negative sentences are facilitated by the presence of a visual context, and demonstrated that this effect persists even when participants only passively view the context.  The contexts provided in Study 1 presumably worked by setting up an expectation that the character in the subsequent trial would have a specific item, making it plausible to negate the absence of that item.  We don't know, however, what contexts are sufficient for setting up such an expectation.  Previous work has found that some contexts are more effective than others at facilitating the processing of negation \cite{ludtke2006, dale2011}, but to our knowledge no studies have explicitly manipulated the strength of the expectation set up by the context.  

Recent work by Frank et al. \cite{frank2012} has attempted to quantify the pragmatic inferences that adults make when playing simple ``language games''. The assumption underlying the predictions made by these authors is that speakers are informative - that is, they will produce utterances that will pick out smaller subsets of the context, leaving as little ambiguity as possible for the listener.  In study 1, the sentence ``Bob has no apples'' is highly informative in reference to a boy with no apples if you have first seen a context in which every boy has apples, because this sentence uniquely identifies the character described in the trial.  However, if only one of the boys in the context had an apple, the sentence ``Bob has no apples'' is less informative because this sentence applies to multiple boys in the context as well.  

In Study 2, we quantitatively manipulated the strength of the context.  In Study 2a, some participants saw contexts in which none of the characters were holding items (i.e. all were empty-handed), some saw contexts in which 1/3  of the characters had items, some saw contexts in which 2/3 of the characters had items, and some saw contexts in which all of the characters had item (as in the context condition in study 1).  Study 2b manipulated contexts in the same way, but contexts showed a total of four people (for a total of five context conditions).  If one thinks of the contexts as giving participants a glimpse into the ``world'' that each trial exists in, the context gives the participant a small sample of the base rate of what the characters in this world look like.  By manipulating the base rate, we can change peoples' expectations about what the trial character will look like.  If the differences in reaction times between the no context and the context condition in study 1 are due to the relative informativeness of the negative utterance based on the context, we should expect to see a linear relationship between the strength of the context and reaction time in response to negative sentences. 

\subsection{Method}

\subsubsection{Participants } 
\emph{Study 2a} 
200 participants were recruited to participate in an online experiment through Amazon's Mechanical Turk website.  Participants ranged in age from 18-70, with the majority of participants being between 18 and 25 (84 participants).  129 participants were male and 71 were female.  Participation was restricted to individuals in the United States, and participants who indicated that their primary language was something other than English were excluded from analysis.  Participants were paid 30 cents to participate in the study, which took approximately 5 minutes to complete.  

\emph{Study 2b}
400 participants were recruited to participate in an online experiment through Amazon's Mechanical Turk website.  Participants ranged in age from 18 to over 65, with the majority of participants being between 18 and 25 years old (184 participants).  205 participants were male and 195 were female.  Participation was restricted to individuals in the United States, and participants who indicated that their primary language was something other than English were excluded from analysis.  Participants were paid 40 cents to participate in the study, which took approximately 7 minutes to complete.

\subsubsection{Stimuli}
\emph{Study 2a} 
Study 2a used the same 28 trial items and sentence types as those used in Study .  A between-subjects ``context'' factor determined what type of context participants saw.  In the ``none'' context condition, participants saw three characters without any objects.  In the ``one'' context condition, participants saw two characters with nothing and once character holding two of the expected object.  In the ``two'' context condition, participants saw one character with nothing and two characters holding the expected objects.  In the ``three'' context condition, participants saw three characters holding the expected objects (identical to the ``Context'' condition in Study 1).  Beneath the three characters was the sentence ``Look at these [boys/girls]!''  Trial stimuli were identical to the stimuli used in Study 1.  

\emph{Study 2b}
In study 2b, the number of items was increased to 48.  The contexts were the same as those in study 2a, except that each context contained 4 boys and there were therefore 5 context conditions (none, one, two, three, and four).  

\subsubsection{Procedure}
 The procedure for study 2a was identical to the procedure in Study 1, with participants randomly assigned to one of the four context conditions.  
 
 In study 2b, participants were given four seconds (instead of three) to look at the context before the experiment automatically advanced.  This was changed to give participants more time to look at the slightly larger contexts; the procedure was otherwise identical to Studies 1 and 2a.  
 
\subsection{Results}


%NOTE: these means are not quite accurate and need to be updated if I decide to post them anywhere!
%\begin{table}[t]
%\caption{\label{tab:e2ameans} Means and standard deviations of reaction times in milliseconds for each of the four trial types, across each of the four contexts (none, one, two, three) for Study 2a (three-person contexts).}
%\begin{center}
%\small\addtolength{\tabcolsep}{-5pt}
%\begin{tabular}{ l  r  r  r  r  r  r  r  r} 
%\hline
%& \multicolumn{2}{c}{None} & \multicolumn{2}{c}{One}  & \multicolumn{2}{c}{Two}  & \multicolumn{2}{c}{Three}\\
%\hline
%  Trial Type & Mean & SD & Mean & SD & Mean & SD & Mean & SD \\ \hline                      
%True Positive & 1519 & 578 & 1412 & 612 & 1456 & 543 & 1487 & 592\\
% False Positive & 1685 & 668 & 1526 & 567 & 1489 & 475 & 1610 & 566\\
% True Negative& 1697 & 593 & 1629 & 666 & 1505 & 498 & 1600 & 560\\
%  False Negative & 1692 & 604 & 1543 & 602 & 1528 & 525 & 1737 & 672\\
%\hline
%\end{tabular}
%\end{center}
%\end{table}
%
%\begin{table}[t]
%\caption{\label{tab:e2bmeans} Means and standard deviations of reaction times in milliseconds for each of the four trial types, across each of the five contexts (none, one, two, three, four) for Study 2b (four-person contexts).}
%\begin{center}
%\small\addtolength{\tabcolsep}{-5pt}
%\begin{tabular}{ l  r  r  r  r  r  r  r  r r r} 
%\hline
%& \multicolumn{2}{c}{None} & \multicolumn{2}{c}{One}  & \multicolumn{2}{c}{Two}  & \multicolumn{2}{c}{Three} & \multicolumn{2}{c}{Four}\\
%\hline
%Trial Type & Mean & SD & Mean & SD & Mean & SD & Mean & SD & Mean & SD \\ \hline 	                
%True Positive & 1519 & 578 & 1412 & 612 & 1456 & 543 & 1487 & 592 & 1487 & 592\\
% False Positive & 1685 & 668 & 1526 & 567 & 1489 & 475 & 1610 & 566 & 1487 & 592\\
% True Negative& 1697 & 593 & 1629 & 666 & 1505 & 498 & 1600 & 560 & 1487 & 592\\
%  False Negative & 1692 & 604 & 1543 & 602 & 1528 & 525 & 1737 & 672 & 1487 & 592\\
%\hline
%\end{tabular}
%\end{center}
%\end{table}

\begin{figure*}
\begin{center} 
\includegraphics[height=1.9in]{figures/study2a_linegraph.pdf}
\includegraphics[height=1.9in]{figures/study2b_linegraph.pdf}
\caption{\label{fig:e2line} Reaction times for each trial type, across different conditions.  Error bars represent the 95\% confidence intervals.  Data for Study 2a (3-person contexts) are shown on the left, and data for Study 2b (4-person contexts) are shown on the right.  }
\end{center} 
\end{figure*}

\begin{figure}
\begin{center} 
\includegraphics[width=3.25in]{figures/combined_plot.pdf}
\caption{\label{fig:e2combined} Combined data for Studies 2a and 2b.  Reaction times for each trial type, across different context proportions (calculated as the proportion of people in the context with the target item).  Error bars represent the 95\% confidence intervals.}
\end{center} 
\end{figure}


\begin{table}[t]
\caption{\label{tab:e2model}Coefficient estimates from a linear effects model estimating the linear effects of context condition, sentence type, and response on reaction time for studies 2a and 2b, accounting for random effects of participant and item.  The data from the two studies is combined, with context included as a continuous variable based on the proportion of characters in the context with the target item}
\begin{center}
\small\addtolength{\tabcolsep}{-5pt}
\begin{tabular}{ r  r  r  r  } 
\hline
  \bf{Fixed Effects} & \bf{Coef.} & \bf{Std. Error} & \bf{t value} \\ \hline        
  Intercept & 1702 & 34 & 49.91\\     
Context(linear) &-167 & 33 & -5.02\\
Sentence Type(Negative) &  42 & 28 & 1.5\\
Truth Value (True) & -151 &   23 &  -6.54\\
Context $\times$ Sentence Type& 29 &   37  &  0.78\\
Context $\times$ Truth Value & 50  &   35  &  1.43\\
Sentence Type $\times$ Truth Value &  162 &   35  &  4.67\\
Context $\times$ Sentence $\times$ Truth Value & -160  &   50.35 &  -3.17\\
\hline
\end{tabular}
\end{center}
\end{table}

Study 2 examined the effect of manipulating the context in ways that should change participants' expectations about what each trial should look like.  We predicted that there would be an inverse linear effect of context, such that as the number of e.g. boys with apples in the context increased, reaction times to evaluate negative sentences would decrease.  

In Study 2a (three-person contexts), seven participants who listed a language other than English as their native language were excluded from analysis.  Fifteen additional participants were excluded for having participated in a previous iteration of the experiment.  Eleven participants were excluded for having an overall accuracy of below 80\%.  Thus, data from a total of 167 participants were analyzed.  

In study 2b (four-person contexts), sixteen participants who listed a language other than English as their native language were excluded from analysis.  Twenty-one additional participants were excluded for having participated in a previous iteration of the experiment.  Twenty-four participants were excluded for having an overall accuracy of below 80\%.  Thus, data from a total of 339 participants were analyzed.  

Only correct trials were analyzed, due to the well-documented speed-accuracy tradeoff found in measures of reaction time.  Trials with RTs greater than 3 standard deviations from the mean were also excluded from the analysis.  A graph of RTs for Studies 2a and 2b can be found in Figure \ref{fig:e2line}.  

Because we were interested in the linear effect of context, results from these two studies were combined and analyzed together, with context condition re-coded as a continuous variable by calculating the proportion of people in each context condition who had a target item (e.g. the one condition in study 2a was recoded as 1/3=.33; the one condition in study 2b was recoded as 1/4=.25, etc.).  A graph of the combined data can be seen in Figure \ref{fig:e2combined}

To examine the reliability of our findings, we fit a linear mixed-effects model to reaction times in response to sentences.  We examined the interaction between sentence type, truth value, and context on reaction times.    Coefficients for this model are shown in Table \ref{tab:e1model}.\footnote{The model specification was as follows: \texttt{RT $\sim$ sentence~$\times$~truth~$\times$~context + (sentence~$\times$~truth~\textbar~subject) +  (sentence~$\times$~truth~\textbar~item)}.}  As in Study 1, we found a significant effect of truth value, with significant faster reaction times for true sentences compared to false sentences ($\beta= -151$, $p< .001$).  Although there was not a significant main effect of negation, there was a significant interaction between sentence type and truth value, such that the difference between true positive and true negative was greater than the difference between the two types of false sentences ($\beta= 162$, $p< .001$).  We also found a linear effect of context on reaction time in line with our predictions, such that as the proportion of people with the target item in the context increased, reaction times decreased ($\beta= -167$, $p< .001$).   There was a significant 3-way interaction between context, sentence type, and truth value, such that the linear effect of context was most striking in true negative sentences ($\beta= -160$, $p< .01$).

Looking at Figure \ref{fig:e2combined}, there appears to be a U-shaped relationship between context and RT, particularly for true negative sentences, such that the ``two'' context has the greatest effect of increasing RTs for these sentences, and RTs for True Negative sentences appear to decrease slightly in the ``all'' context condition.  To test the significance of this curve, we added a term to our model to test for the quadratic effect of context.  This model was a significantly better fit to our data ($\chi^{2}(1) = 71.25$, $p<.001$).  This model reported a significant quadratic effect of context,  ($\beta= 696 $, $p< .001$).  ((NOTE: I'm not sure exactly how I should report these results.  Maybe I should discuss the u-shaped relationship earlier and then just report the results of the model with the quadratic term?  Everything else stays basically the same, so it wouldn't change the interpretation, just how the results are presented))

\subsection{Discussion}
Study 2 was designed to test how the effect of context changes when the strength of the context is quantitatively manipulated.  We hypothesized that as the strength of the context increased (i.e. the number of boys with apples in the context), reaction times would decrease.  We found a significant linear effect of context, consistent with our prediction.  We also found a significant quadratic effect of context.  Looking at the data, it appears that the contexts that had the most significant effect on processing time were those in which 60\%-75\% of the characters in the context had the target item; contexts in which all of the characters had the target item lead to slightly increased reaction times.    

The results of Study 2 support our hypothesis that quantitatively manipulating the strength of the context results in systematic changes in sentence processing, particularly for true negative sentences.  We proposed that this effect is due to the ways that context influences how informative a sentence is about a given picture.  Our results are consistent with this finding.  As the proportion of people in the context with the target item increases, describing the trial picture as \emph{not} having that target item becomes more informative.  That is, the more people in the context who have e.g. apples, the more we expect a person with nothing to be described as ``a boy with no apples''.  According to this prediction, as the context changes, our expectations of how a speaker would describe a picture should also change, and these predictions should be related to how quickly people process these sentences.  Study 3 tests this prediction experimentally.  

\section{Study 3: Measuring Listener Expectations}
In studies 1 and 2, we demonstrated that a simple visual context can facilitate the processing of negative sentences, and that there is a linear (quadratic?) relationship between the strength of the context and the effect on reaction time to evaluate negative sentences.  Contexts that set up a strong expectation lead to faster reaction times to process a negative sentence when that expectation is violated.  Our prediction was based on previous work demonstrating that participants expect speakers to be informative when speaking \cite{frank2012}.  If everyone in the context has a specific feature, and the trial character is lacking that feature, it is highly informative to describe the trial character in terms of the negation of the expected feature.  However, although the results of Studies 1 and 2 support this interpretation, we do not know for sure whether participants expect speakers to use negation in these contexts.  Study 3 attempts to directly measure participants' expectations of how a speaker would describe the pictures seen in Studies 1 and 2, depending on the context.

Our hypothesis here draws on the idea of \emph{surprisal}, a information-theoretic measure of the amount of information contained in an utterance.  Previous work has proposed that the processing difficulty of a word is the surprisal of that word, calculated as the -log of the probability of the word occurring in a given syntactic and semantic context \cite{levy2008}.  This theory proposes that expectations about what utterance a speaker will produce are related to a listener's speed to evaluate that sentence.  We draw on this theory in the realm of pragmatics, developing pragmatic theory of surprisal to explain the effect of context on the processing of negative sentences.  

To test this theory, we conducted a study to evaluate participants' expectations of how a speaker would describe a picture in a given context.  Participants were introduced to an imaginary speaker who was described as either ``an honest guy who says things that are plausible and true'' or ``a tricky guy who says things that are plausible but false''.  This was originally set up to test participants expectations about false sentences; however, participants in the ``tricky'' condition struggled to understand this task, and results from this condition are not presented here.  Participants were then shown a set of three or four characters that the speaker ostensibly also saw (the contexts from the previous studies), and then shown a single character and asked to bet whether the speaker would use different sentences to describe the picture.  This allowed us to calculate the surprisal of positive and negative sentences based on participants' expectations about what sentences a speaker would use to describe these pictures, and determine if a relationship exists between the pragmatic surprisal of a sentence and reaction times to evaluate those sentences.  

\subsection{Method}

\subsubsection{Participants}
280 participants were recruited to participate in an online experiment through Amazon's Mechanical Turk website.  Participants ranged in age from 18 - 65+, with the majority of participants being between 25 and 35 (109 participants).  167 participants were male and 113 were female.  Participation was restricted to individuals in the United States, and participants who indicated that their primary language was something other than English were excluded from analysis.  Participants were paid 30 cents to participate in the study, which took approximately 5 minutes to complete.  

\subsubsection{Stimuli}
A subset of 12 items out of the items used in Studies 1 and 2 were used for this study.  As in the previous studies, these items consisted of a context which presented either three (Study 3a) or four (Study 3b) characters who were either holding two of the same item or holding nothing, and a trial character who was presented alone holding either two items or holding nothing.  

On each trial, participants were given four sentences to bet on.  The four sentences were always a positive sentence involving the target item, a negative sentence involving the target item, a positive sentence involving an alternative item, and a negative sentence involving an alternative item.  This meant that for pictures of a person holding a target item (e.g. a picture of a boy with apples), the two true sentence options were the positive-item sentence and the negative-alternative sentence (e.g. ``A boy with apples'' and ``A boy with no cookies'').  For pictures of a person holding nothing, the two true sentence options were the negative-item sentence and the negative-alternative sentence (e.g. ``A boy with no apples'' and ``A boy with cookies'').   

\subsubsection{Procedure}
As in Studies 1 and 2, participants were first presented with an instructions screen which described the task and informed them that they would be participating in a psychology experiment and could stop at any time.  Once participants accepted the task, they were introduced to the speaker, Joe, who was described as ``an honest guy who says things that are plausible and true''.  They were told that they would see different pictures, and their job was to decide how likely Joe is to use different sentences to describe the picture.  Participants were told that they could bet on more than one sentence, but that their bets must sum to 100.  

Participants were then given two trials to practice betting.  Practice trials always showed a person with items, and did not include context slides.  Participants were reminded of the task on each trial.  During practice trials, participants were only given positive sentences to choose from, and only one of the sentences was true.  If participants bet on a false sentence, they were reminded that the speaker is an honest person who says things that are true.  This feedback was not given during test trials.

Experiment trials differed from practice trials in that participants first viewed a context slide similar to the context slides presented in Studies 1 and 2.  Participants saw three (Study 3a) or four (Study 3b) characters with either none, one, two, three, or four of the characters holding target items with the rest of the characters holding nothing.  Beneath the characters, participants were told ``Joe sees these [boys/girls]''.  The context screen was displayed for four seconds and then the experiment automatically proceeded to the test trial for that item.  

Test trials showed a single boy or girl either holding the same items or holding nothing.  Under the picture were four sentences (positive-item, negative-item, positive-alternative, and negative-alternative).  Participants were told to place bets on whether Joe would use these sentences to describe this picture, and were reminded that their bets must sum to 100.  

\subsection{Results}
\begin{figure}
\begin{center} 
\includegraphics[width=3.25in]{figures/speakerstudy_comparison.pdf}
\caption{\label{fig:e3plot} A comparison of surprisal from Study 3 (calculated as the -log of the mean bet on each sentence type) and reaction times from Study 2.  Error bars represent the 95\% confidence intervals.  }
\end{center} 
\end{figure}

Study 3 was designed to test the effect of context on people's expectations of how a speaker would describe a picture.  We predicted that there would be a correlation between participants predictions of what a speaker would say to describe a picture, and reaction times to evaluate the predicted sentences.  

Eleven participants who listed a language other than English as their native language were excluded from analysis.  Forty-six additional participants were excluded for having participated in a previous iteration of the experiment. Thus, data from a total of 223 participants were analyzed.  

In each trial, participants had the ability to bet on four possible sentences: a positive sentence about the target item, a negative sentence about the target item, a positive sentence about an alternative item, and a negative sentence about the alternative item.  Any bets on sentences that were logically false were excluded; very few participants bet on false sentences (the average participant allocated less than 5\% of their bets to false sentences).  This eliminated all bets to alternative-positive sentences (e.g. ``A boy with cookies'' when the target item was apples), because these sentences were always false.  We also eliminated all bets to alternative-negative items (e.g. ``A boy with no cookies'' when the target item was apples).  Although these sentences were sometimes true (e.g. when the trial picture showed a boy with nothing), we were interested in seeing how participants bets on item negative sentences changed as the context changed.  Thus, we only analyzed bets to item-positive sentences when the trial picture showed a character holding the target items (e.g. bets on ``A boy with apples'' when the picture showed a boy holding apples) and bets to item-negative sentences when the trial picture showed a character holding nothing (e.g. bets on ``A boy with no apples'' when the picture showed a boy holding nothing).  These were analogous to the True Positive and True Negative sentences in Studies 1 and 2, allowing us to compare expectations about what a speaker would say to participants' reaction times in Study 2.  

We used participants' expectations about how a speaker would describe a given picture to calculate surprisal, an information-theoretic measure that previous work has suggested is related to the processing difficulty of a word \cite{levy2008}.  Surprisal is calculated as the -log of the probability of a word given the context.  We calculated the probability of a word given the context by taking the mean bet for each sentence type (true negative and true positive, as described above) for each context condition, and took the -log of these mean bets to compute the surprisal of each sentence type for each context condition.  We then compared this measure of surprisal to the mean reaction times for true positive and true negative sentences for each context condition, as measured in Study 2.  A graph of this comparison can be seen in Figure \ref{fig:e3plot}.  There was a significant correlation between surprisal and reaction time in these data ($r=.82$, $p<.001$).  

\subsection{Discussion}

In Studies 1 and 2, we found that including a visual contexts facilitates the processing of negative sentences, and that this effect is modulated by the strength of the expectation set up by the context.  As the proportion of people with a target item in the context increases, the reaction time to process a negative sentence that negates that target property decreases.  We postulated that this effect is due to the ways that context changes expectations about what a speaker might say to describe a picture.  
If you see a context in which the majority of boys are holding apples, and then see a boy holding nothing, it makes sense to predict that the boy holding nothing might be described as a ``boy with no apples''.  However, if you see a context in which nobody is holding anything, it would be odd to describe another person holding nothing as ``a boy with no apples'', because you have no reason to expect that he \emph{would} have apples.  Our prediction was that these expectations, calculated as the surprisal of seeing a picture described using a certain sentence, would be positively correlated with reaction times from previous studies.  Study 3 confirmed this hypothesis.  

Why is it that different contexts influence our predictions of how a speaker might describe a picture?  Our initial hypothesis was built on work by Frank and Goodman (2012), who created a model of language comprehension based on the Grice's Maxim of Quantity, which dictates that speakers should choose utterances that are maximally informative.  This model demonstrated that listeners expect speakers to produce utterances that are highly informative.  With respect to this work, the utterance ``A boy with no apples'' is highly informative if it refers to a boy with nothing in a world where every other boy has apples, because this utterance uniquely identifies the intended referent.  However, in a world where none of the other boys are holding apples, the sentence ``A boy with no apples'' is uninformative, because it refers to every other boy in this hypothetical world.  In the next section, we test whether a model of informativeness can predict processing times for true positive and true negative sentences.  

\section{Model}
Sections for model???

\begin{figure}
\begin{center} 
\includegraphics[width=3.25in]{figures/model1_comparison.pdf}
\caption{\label{fig:addition_subs} A comparison of model 1 predictions and reaction times in Study 2.  Error bars represent the 95\% confidence intervals.}
\end{center} 
\end{figure}

\begin{figure}
\begin{center} 
\includegraphics[width=3.25in]{figures/model2_comparison.pdf}
\caption{\label{fig:addition_subs} A comparison of model 2 predictions and reaction times in Study 2.  Error bars represent the 95\% confidence intervals.}
\end{center} 
\end{figure}


\section{General Discussion}
Ignore this...it is the conclusion I wrote for this paper over a year ago.  I haven't even read it, I just copied and pasted it here, and it is probably terrible.  

This paper explores the relationship between the processing of negative sentences, and the contexts in which these sentences are presented.  Previous work has shown that although it takes longer to evaluate negative sentences than positive sentences when test sentences are presented with no context \cite{carpenter1975, just1971, just1976, hclark1972}, the presence of a verbal context \cite{dale2011, glenberg1999, ludtke2006} or the process of describing a visual context \cite{wason1965} can facilitate the processing of negative sentences.  This suggests that a pragmatic account of negative sentences is warranted, such that negative sentences are licensed in contexts which set up an expectation which is then violated.  However, the extent to which expectation modulates the processing of negative sentences has not been systematically examined until now.  

In the studies presented here, we quantitatively manipulated the expectation set up by the context by changing the proportion of people in the context who held a certain target item.  We found a quadratic relationship between the proportion of people in the context with the target item and the reaction times to respond to negative sentences.  This effect was most striking in response to true negative sentences, in which participants saw a person holding nothing and saw a sentence such as ``Bob has no apples''.  True positive sentences showed little, if any, effect of context.  

To understand the quadratic effect of context, we must consider how the context alters a participant's expectations about the picture that may appear as well as the likelihood of a negative term being used to describe that pictures.  In our original predictions, we considered only the effects of expectation on the likelihood that a negative would be used to describe the picture; that is, the stronger the expectation that e.g. boys have apples, the more likely a negative will be used to describe a picture of a boy with no apples.  However, a linear prediction does not account for the surprisal incurred by seeing a boy with no apples when you've received a context that suggests that \emph{all} boys have apples.  If this is taken into account, the relationship between context and reaction time seen in studies 2 and 3 makes sense - reaction times decrease as the proportion of people with the target item increases, but reaction times to negative sentence increase in the ``all'' context due to surprise at seeing an unexpected boy or girl with nothing.  (I'm just repeating myself now, and not saying anything interesting, so I'm going to stop and reconsider what I want to be saying here as I wrap up this paper...)


possibly cite pea 1979?



\section{Acknowledgments}
This material is based upon work supported by the National Science Foundation Graduate Research Fellowship. Any opinion, findings, and conclusions or recommendations expressed in this material are those of the authors(s) and do not necessarily reflect the views of the National Science Foundation.


\bibliographystyle{apacite}

\setlength{\bibleftmargin}{.125in}
\setlength{\bibindent}{-\bibleftmargin}

\bibliography{bibLibrary}

\end{document}

